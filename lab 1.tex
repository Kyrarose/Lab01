% Digital Logic Report Template
% Created: 2020-01-10, John Miller

%==========================================================
%=========== Document Setup  ==============================

% Formatting defined by class file
\documentclass[11pt]{article}

% ---- Document formatting ----
\usepackage[margin=1in]{geometry}	% Narrower margins
\usepackage{booktabs}				% Nice formatting of tables
\usepackage{graphicx}				% Ability to include graphics

%\setlength\parindent{0pt}	% Do not indent first line of paragraphs 
\usepackage[parfill]{parskip}		% Line space b/w paragraphs
%	parfill option prevents last line of pgrph from being fully justified

% Parskip package adds too much space around titles, fix with this
\RequirePackage{titlesec}
\titlespacing\section{0pt}{8pt plus 4pt minus 2pt}{3pt plus 2pt minus 2pt}
\titlespacing\subsection{0pt}{4pt plus 4pt minus 2pt}{-2pt plus 2pt minus 2pt}
\titlespacing\subsubsection{0pt}{2pt plus 4pt minus 2pt}{-6pt plus 2pt minus 2pt}

% ---- Hyperlinks ----
\usepackage[colorlinks=true,urlcolor=blue]{hyperref}	% For URL's. Automatically links internal references.

% ---- Code listings ----
\usepackage{listings} 					% Nice code layout and inclusion
\usepackage[usenames,dvipsnames]{xcolor}	% Colors (needs to be defined before using colors)

% Define custom colors for listings
\definecolor{listinggray}{gray}{0.98}		% Listings background color
\definecolor{rulegray}{gray}{0.7}			% Listings rule/frame color

% Style for Verilog
\lstdefinestyle{Verilog}{
	language=Verilog,					% Verilog
	backgroundcolor=\color{listinggray},	% light gray background
	rulecolor=\color{blue}, 			% blue frame lines
	frame=tb,							% lines above & below
	linewidth=\columnwidth, 			% set line width
	basicstyle=\small\ttfamily,	% basic font style that is used for the code	
	breaklines=true, 					% allow breaking across columns/pages
	tabsize=3,							% set tab size
	commentstyle=\color{gray},	% comments in italic 
	stringstyle=\upshape,				% strings are printed in normal font
	showspaces=false,					% don't underscore spaces
}

% How to use: \Verilog[listing_options]{file}
\newcommand{\Verilog}[2][]{%
	\lstinputlisting[style=Verilog,#1]{#2}
}




%======================================================
%=========== Body  ====================================
\begin{document}

\title{ELC 2137 Lab 01: Git and LaTex Intro}
\author{Kyra Rose}

\maketitle


\section*{Summary}

The purpose of this lab was to get an introduction to GitHub and coding using LaTeX, describe the purpose of LaTex and GitHub, and to create a lab report demonstrating basic knowledge of LaTeX.  


\section*{Q\&A}

1) What is your GitHub user name?

	Kyra\_rose1

2) What LaTeX environment produces bulleted (non-numbered) list?

	An itemized environment.

3) Write the equation y(t) = 1/2e-t using LaTeX equation formatting.

	$y(t)=1/2*e^-t$

4) What is the shortcut key for compiling your LaTeX document?

	F5

\section*{Code}

\Verilog{lab1_example_code.sv}

\section*{Results}

\begin{figure}[ht]\centering
	\begin{tabular}{l|rrrr}
		Binary & Hex & Decimal \\
		\midrule
		0000 & 0 & 0 \\
		0010 & 2 & 2 \\
		0100 & 4 & 4 \\
		0110 & 6 & 6 \\
		1000 & 8 & 8 \\
		1010 & A & 10 \\
		\bottomrule
	\end{tabular}\medskip
	
	\includegraphics[trim=7.3in 6.2in .2in 1.75in,clip,width=6in]{lab1_example_screenshot.PNG}
	\caption{Example of a simulation waveform and ERT}
	\label{fig:sim_with_table}
	
	\includegraphics[trim= 6in 5in 8.1in 5.3in,clip,width=7in,height=5in]{GitHub History Screenshot}
	
\end{figure}

\end{document}
